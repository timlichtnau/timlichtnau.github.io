\documentclass[12pt,headings=small,paper=A4,DIV=calc]{article}

\title{Covering Letter \\ Application for PhD Position in Nottingham}
\author{Tim Lichtnau}
\usepackage{amsmath}
\usepackage{amssymb}
\usepackage{hyperref}

\begin{document}
\maketitle
My name is Tim Lichtnau, I am 23 years old and I am a master's student at the University of Bonn. Even in kindergarten, I knew that I wanted to become a mathematician. Luckily, I grew up in a highly inspiring environment. On one hand, I participated in nation-wide training for math olympiads (\emph{Jugend trainiert Mathematik}), which brought me silver at the national level in the Math olympiad and to the final round of the Bundeswettbewerb, but more importantly it showed me that math is a team sport: \emph{Jugend trainiert Mathematik} taught me to discuss mathematical ideas with people across the country and make friends sharing the same passion. On the other hand, from the age of 12, I received individual mentoring from Professor Ines Kath in my hometown Greifswald once a week. Together, we explored topics like induction over the naturals, analysis, Galois theory, and later topology. 

This strong foundation enabled a smooth start to my studies in mathematics at Bonn, where I focused on algebraic topology, algebra and computer science as my secondary subject. In my second term, I began reading about category theory---unfortunately, there were no lectures on this topic in Bonn. From that point, category theory became my constant companion. \\
In my fourth term, I co-authored a \href{https://ceur-ws.org/Vol-3377/natfom2.pdf}{paper} on the Naproche system (Natural Proof Checking) which processes input texts approximating the natural language of mathematics. The paper was published for the Computational Intelligence of Computer Mathematics conference, where I gave a talk as well. Even then, my niche in mathematics was emerging: somewhere between topology and the formalization of mathematics. Meanwhile, I became a teaching assistant and continued in that role for the remainder of my time in Bonn. I realized that teaching is very important to me. I want to transport the passion I have for the topic to the students. I got proposed for the Studienstiftung by the university, because judging by the grades I was one of the best 4 students of my cohort.  

For my \href{https://uni-bonn.sciebo.de/s/PJUqHE52SnCKJMy}{bachelor's thesis}, I began studying higher category theory, which opened an entirely new world for me. During my master's, while participating in a reading groups on higher category theory, I focused mainly on algebraic geometry and algebraic topology. I took a look in Computational Topology as well. Moreover, I organized a 1-year informal \href{https://uni-bonn.sciebo.de/s/eEHziabNePFwvb4}{seminar} with Jonathan Wassermann to compensate for the lack of a category theory lecture in Bonn. I wrote \href{https://uni-bonn.sciebo.de/s/RzjF14df9WvD3O5}{lecture notes} for this seminar. Teaching was incredibly rewarding, and the positive feedback of the large audience we received confirmed its success. 

I felt that some areas of mathematics were taught using unnecessarily complicated language---like working with one hand tied behind your back. For me, the set-theoretic framework was not a satisfying foundation for the fields I was most interested in. \\
Two key events shifted my perspective further. First, I learned about Homotopy Type Theory (HoTT) and Agda at the HoTTest Summer School. It was one of the most beautiful intellectual experience of my life. From that moment, I knew what I wanted to do: make the lives of mathematicians easier with HoTT. I read most of the HoTT book and Egbert Rijke's book on the subject. 
Second, building on this foundation, I attended workshops where I discovered \href{https://github.com/felixwellen/synthetic-zariski}{Synthetic Algebraic Geometry}, which allows one to work internally in the Zariski $\infty$-topos rather than introducing a complicated theoretical framework to discuss structure sheaves or generic points. Grothendieck might have appreciated this synthetic approach, as the proofs are clearer and more elementary than those in conventional theory. I organized with Heiko Braun a one-week course at the QED-club on HoTT and Synthetic Algebraic Geometry, contrasting it with the often arduos process of learning algebraic geometry at university. \\
Next to my favorite proof assistant Agda, I studied Lean under Floris van Doorn. I formalized the \href{https://github.com/timlichtnau/LeanCourse23/tree/master/LeanCourse/Project}{fibered Yoneda lemma} and, as part of a \href{https://github.com/timlichtnau/BonnAnalysis/tree/master/BonnAnalysis}{collaborative project}, formalized distribution theory. I also participated in the Trimester Program on Formal Mathematics in Bonn and attended the ``From Analysis to Homotopy Theory'' conference in Greifswald. In addition to my academic pursuits, I developed software in Haskell for the startup \href{https://www.digitallyinduced.com/}{Digitally Induced}. 

Then Hugo Moeneclaey invited me to supervise my \href{https://github.com/timlichtnau/MasterThesis/blob/Main}{master's thesis} on Synthetic Algebraic Geometry. My examiner from Bonn is Floris van Doorn. My work focuses on the synthetic approachh to algebraic stacks. Classically, it is very demanding to work with Algebraic Stacks, because they require a lot of machinery: ``algebraic stacks and the geometry required to define them`` are documented in the Stacks Project, a massive body of thousands of pages. One goal of the thesis is to find an easier more direct approach to work with Algebraic Stacks. Furthermore, through my work I am getting more and more confident with lex modalities in HoTT. 

Also, as a guest student of the University of Gothenburg, I am enjoying the academic environment of the Type Theory research group. For example, I participate in a reading group organized by Christian Sattler, where we work on constructivizing Cisinski's book on higher categories and homotopical algebra, that is related to the topic of a reading group I participated in Bonn about the Formalization of Higher category theory (Cisinski). The synthetic approach deeply resonates with me: I wish to present mathematics easier, closer to the foundation, so that one sees what is really going on instead of beeing distracted by the formalism and the choice of model. \\
In my free time, I love thinking about private projects: 
\begin{itemize}
	\item I work one something that one could call synthetic Homological Algebra: Given any abelian category $\mathcal{A}$, we associate a category of spans, where a morphism $A \rightsquigarrow B$ is given by a span $A \twoheadleftarrow C \to B$ in $\mathcal{A}$. We think of such a $A \rightsquigarrow B$ as a way of producing a (possibly non well-defined) term of $B$ from a term of $A$. By the universal property of the quotient, such a $A \rightsquigarrow B$ comes from an actual morphism $A \to B$, iff $0_A$ is \emph{necessarily} send to $0_B$. I formalized such synthetic diagram chases in \href{https://github.com/timlichtnau/CarpetsOnAgda/blob/master/}{Agda}, without mentioning a single time the $\mathsf{Ab}$-enrichment. One can equip these ideas with a \href{https://www.3blue1brown.com/content/blog/exact-sequence-picturebook/PuzzlingThroughExactSequences.pdf}{frontend} suggested by Ravi Vakil which makes it easy to do diagram chases and proving the snake lemma. 
	\item  I love to think about category theory. Currently, I am developing a calculus for canonical morphisms, aiming to answer the question: is there a canonical morphism $A \to B$ in a certain context, and if so, how does it look like? This work is grounded in the framework of fibered categories, and I have received helpful feedback from Paul-André Melliès and Denis-Charles Cisinski. 
\end{itemize}
\textbf{Contact details of my supervisors:} \\
Thierry Coquand: \texttt{Thierry.Coquand@cse.gu.se}, Floris van Doorn: \texttt{fpvdoorn@gmail.com}
\end{document}


\end{document}